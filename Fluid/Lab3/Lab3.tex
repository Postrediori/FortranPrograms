\documentclass[12pt,a4paper,oneside]{book}

\usepackage[utf8x]{inputenc}
\usepackage[T2A]{fontenc}
\usepackage[russian]{babel}


\usepackage{ucs}
\usepackage{cmap}
\usepackage{amsmath}
\usepackage{amsthm}
\usepackage{mathtools}

\makeatletter
  \newcommand*\Mach{\mathcal{M}}
  \newcommand*\crit{\text{кр}}
  \newcommand*\eps{\varepsilon}
\makeatother

\title{Лабораторная работа №3. \\ Определение параметров потока течения Прандтля-Майера}
\author{}
\date{}

\begin{document}
  
  \maketitle
  
  \section*{Задание}
  
  1. Разработать подпрограммы, позволяющие определить число Маха $\Mach_2$
  по заданным значением числа $\Mach_1$ и угла $\theta$. Применить для
  этого приближённые методы.
  
\end{document}
