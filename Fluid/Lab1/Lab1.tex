\documentclass[12pt,a4paper,oneside]{book}

\usepackage[utf8x]{inputenc}
\usepackage[T2A]{fontenc}
\usepackage[russian]{babel}


\usepackage{ucs}
\usepackage{cmap}
\usepackage{amsmath}
\usepackage{amsthm}
\usepackage{mathtools}

\makeatletter
  \newcommand*\Mach{\mathcal{M}}
  \newcommand*\crit{\text{кр}}
  \newcommand*\eps{\varepsilon}
\makeatother

\title{Лабораторная работа №1. \\ Свойства основных газодинамических функций}
\author{}
\date{}

\begin{document}
  
  \maketitle
  
  При решении газодинамических задач удобно использовать основные
  газодинамические функции, для которых составлены таблицы. Однако, применение
  таблиц не всегда удобно, так как часто требует проведения интерполирования,
  особенно при нахождении значения аргумента по заданному значению функции
  (т.е. при вычислении обратной функции). Современное развитие вычислительной
  техники и её доступность позволяют отказаться от использования таблиц.
  
  Целью лабораторной работы является разработка программы, позволяющей
  вычислять значения основных газодинамических, а также соответствующие
  обратные функции.
  
  В качестве аргумента газодинамических функций можно использовать либо
  число Маха $\Mach$, либо коэффициент скорости $\lambda$. Эти
  параметры связаны соотношениями
  
  \begin{equation*}
    \Mach^2 = \frac{2 \lambda^2}{(k + 1) - (k - 1) \lambda^2},
    \lambda^2 = \frac{(k + 1) \Mach^2}{2 + (k - 1) \Mach^2}
  \end{equation*}
  
  \noindent с помощью которых можно один параметр выразить через другой
  и наоборот.
  
  Основные газодинамические функции следующие:
  
  \begin{align*}
    & \tau = \frac{T}{T_0}
      = \Big( 1 + \frac{k - 1}{2} \Mach^2 \Big)^{-1}
      = 1 - \frac{k - 1}{k + 1} \lambda^2 \\
    & \pi = \frac{p}{p_0}
      = \Big( 1 + \frac{k - 1}{2} \Mach^2 \Big)^{\frac{-k}{k - 1}}
      = \Big( 1 - \frac{k - 1}{k + 1} \lambda^2 \Big)^{\frac{k}{k - 1}} \\
    & \eps = \frac{\rho}{\rho_0}
      = \Big( 1 + \frac{k - 1}{2} \Mach^2 \Big)^{\frac{-1}{k - 1}}
      = \Big( 1 - \frac{k - 1}{k + 1} \lambda^2 \Big)^{\frac{1}{k - 1}} \\
    & q = \frac{F_{\crit}}{F}
      = \frac{\rho V}{\rho_{\crit} a_{\crit}}
      = \lambda \Big( 1 - \frac{k - 1}{k + 1} \lambda^2 \Big)^{\frac{1}{k - 1}}
        \Big( \frac{2}{k + 1} \Big)^{\frac{-1}{k - 1}}
  \end{align*}
  
  \noindent Здесь $p$ -- давление, $\rho$ -- плотность, $T$ -- температура,
  $k$ -- показатель адиабаты.
  
  Между этими функцийми существуют зависимости
  
  \begin{equation*}
    \tau = \pi^{\frac{k - 1}{k}} = \eps^{k - 1},
    q = \frac{\lambda \eps}{\eps (1)}
  \end{equation*}
  
  \section*{Задание на лабораторную работу}
  
  1. Разработать программу для вычисления значений функций $\eps$, $\pi$,
  $\tau$, $q$ по заданному значению $\lambda$ или $\Mach$.
  
  2. Построить графики газодинамических функций в виде зависимости от $\lambda$,
  где $\lambda$ изменяется от $0$ до $\sqrt{\frac{k + 1}{k - 1}}$.
  
  3. Построить зависимости критических значений функций $\eps(1)$, $\pi(1)$,
  $\tau(1)$ от $k$, где $k$ изменяется от $1,1$ до $1,67$.
  
  4. Методом итераций построить обратную функций для $q(\lambda)$, учитывая
  её двузначность.
  
\end{document}
